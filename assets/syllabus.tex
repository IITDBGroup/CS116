\documentclass[12pt]{article}
\textwidth=7in
\textheight=9.5in
\topmargin=-1in
\headheight=0in
\headsep=.5in
\hoffset  -.85in

\usepackage{color}
\usepackage{hyperref}
\pagestyle{empty}

\renewcommand{\thefootnote}{\fnsymbol{footnote}}
\begin{document}

\begin{center}
{\bf \Large CS 116 Introduction to Object-Oriented Programming II}\\
%{\bf 3:15 pm - 4:30 pm, Mondays + Wednesdays, SB 104}
\end{center}

\setlength{\unitlength}{1in}

\begin{picture}(6,.1)
\put(0,0) {\line(1,0){6.25}}
\end{picture}

\renewcommand{\arraystretch}{2}
%%%%%%%%%%%%%%%%%%%%%%%%%%%%%%%%%%%%%%%%%%%%%%%%%%%%%%%%%%%%
\vskip.25in
{\large \noindent\textbf{Instructor:}} Boris Glavic, Email: \verb!bglavic@iit.edu!

\vskip.25in
{\large \noindent\textbf{Instructor Webpage:}} \url{www.cs.iit.edu/~glavic}

\vskip.25in
{\large \noindent\textbf{Course Webpage:}} \url{www.cs.iit.edu/~glavic/cs116}

\begin{picture}(6,.1)
\put(0,0) {\line(1,0){6.25}}
\end{picture}

%%%%%%%%%%%%%%%%%%%%%%%%%%%%%%%%%%%%%%%%%%%%%%%%%%%%%%%%%%%%
\vspace*{.15in}

{\large \noindent \textbf{Course Description} }

Continuation of CS 115. Introduces more advanced elements of object-oriented programming – including dynamic data structures, recursion, searching and sorting, and advanced object-oriented programming techniques. For students in CS and CS related degree programs. \\

2 credit hours; required for CS \& CPE (or CS201); 75 min. lecture \& 75 min. lab each week\\


%%%%%%%%%%%%%%%%%%%%%%%%%%%%%%%%%%%%%%%%%%%%%%%%%%%%%%%%%%%%
\vskip.25in
{\large \noindent\textbf{Course Material}  }

\noindent The following text book is required reading material for the course.\\

{\footnotesize
  \noindent Julie Anderson and Herv\'e J. Franceschi, \textbf{Java Illuminated}, Jones \& Bartlett Learning, \textbf{ISBN-13:} 978-1284140996
}\\[3pt]

\noindent Other good more advanced Java books are:\\

{\footnotesize
\noindent   Joshua Bloch, \textbf{Effective Java}, Addison Wesley, \textbf{ISBN-13:} 978-0134685991
}\\[3pt]

\noindent Notebooks, labs, and slides are made available through the course webpage


%%%%%%%%%%%%%%%%%%%%%%%%%%%%%%%%%%%%%%%%%%%%%%%%%%%%%%%%%%%%
\vskip.25in
{\large \noindent\textbf{Prerequisites:}}

\begin{itemize}
\item \textit{Courses:} CS 115 (2-1-2)
\end{itemize}


%%%%%%%%%%%%%%%%%%%%%%%%%%%%%%%%%%%%%%%%%%%%%%%%%%%%%%%%%%%%
\vskip.25in
{\large \noindent\textbf{Students with Disabilities}}

Reasonable accommodations will be made for students with documented disabilities. In order to receive accommodations, students must obtain a letter of accommodation from the Center for Disability Resources. The Center for Disability Resources (CDR) is located in 3424 S. State St., room 1C3-2 (on the first floor), telephone 312 567.5744 or \url{disabilities@iit.edu}.

%%%%%%%%%%%%%%%%%%%%%%%%%%%%%%%%%%%%%%%%%%%%%%%%%%%%%%%%%%%%
\vskip.25in
{\large \noindent\textbf{Piazza}}

his term we will be using Piazza for class discussion. The system is highly catered to getting you help fast and efficiently from classmates, the TA, and myself. Rather than emailing questions to the teaching staff, I encourage you to post your questions on Piazza. If you have any problems or feedback for the developers, email team@piazza.com.

The link to the Piazza webpage can be found here: \url{http://cs.iit.edu/~glavic/cs116}



\pagebreak


%%%%%%%%%%%%%%%%%%%%%%%%%%%%%%%%%%%%%%%%%%%%%%%%%%%%%%%%%%%%
\vskip.25in
{\large \noindent \textbf{Course Objectives} }

Students should be able to:

\begin{itemize}
\item Analyze and explain the behavior of simple programs involving the following fundamental programming constructs: assignment, I/O (including file I/O), selection, iteration, functions, pointers
\item Write a program that uses each of the following fundamental programming constructs: assignment, I/O (including file I/O), selection, iteration, functions, pointers
\item Break a problem into logical pieces that can be solved (programmed) independently.
\item Develop, and analyze, algorithms for solving simple problems.
\item Use a suitable programming language, and development environment, to implement, test, and debug algorithms for solving simple problems.
\item Write programs that use each of the following data structures (and describe how they are represented in memory): strings, arrays, structures, and class libraries including strings and vectors
\item Explain the basics of the concept of recursion.
\item Write, test, and debug simple recursive functions and procedures.
\item Explain and apply object-oriented design and testing involving the following concepts: data abstraction, encapsulation, information hiding, sub-classing, inheritance, templates
\item Use a development environment to design, code, test, and debug simple programs, including multi-file source projects, in an object-oriented programming language.
\item Solve problems by creating and using sequential search, binary search, and quadratic sorting algorithms (selection, insertion)
\item Determine the time complexity of simple algorithms.
\end{itemize}

%%%%%%%%%%%%%%%%%%%%%%%%%%%%%%%%%%%%%%%%%%%%%%%%%%%%%%%%%%%%
\pagebreak
{\large \noindent \textbf{Course Topics} }

\noindent The following topics will be covered in the course:

\begin{itemize}
\item The Java language
  \begin{itemize}
  \item Java's type system and syntax
  \item The Java class library
  \item Documentation
  \item Exception Handling
  \end{itemize}
\item Object-oriented Programming and Design
  \begin{itemize}
  \item Classes, Instances, Inheritance, and Polymorphism
  \item Design Patterns
  \end{itemize}
\item Debugging, Logging, and Testing
\item Data Structures
  \begin{itemize}
  \item Lists
  \item Maps
  \item Trees
  \end{itemize}
\item Basic Algorithms
  \begin{itemize}
  \item Searching and Sorting
  \item Computational Complexity and Runtime Analysis
  \end{itemize}
\item Programming Techniques
  \begin{itemize}
  \item Recursion
  \item Concurrent Programming
  \end{itemize}
\item Tool ecosystem
  \begin{itemize}
  \item Editors and IDEs
  \item Java binaries
  \item Version Control
  \end{itemize}
\end{itemize}

\pagebreak
%%%%%%%%%%%%%%%%%%%%%%%%%%%%%%%%%%%%%%%%%%%%%%%%%%%%%%%%%%%%
\vskip.25in
{\large \noindent \textbf{Workload and Grading Policies:} }


%%%%%%%%%%%%%%%%%%%%%%%%%%%%%%%%%%%%%%%%%%%%%%%%%%%%%%%%%%%%
\vskip.25in
\noindent\textbf{Course Project}:

\noindent There will be a programming project in the later part of the course.


%%%%%%%%%%%%%%%%%%%%%%%%%%%%%%%%%%%%%%%%%%%%%%%%%%%%%%%%%%%%
\vskip.25in
\noindent\textbf{Midterm and Final Exam}:

\noindent There will be a midterm and final exam covering the topics of the course.

%%%%%%%%%%%%%%%%%%%%%%%%%%%%%%%%%%%%%%%%%%%%%%%%%%%%%%%%%%%%
\vskip.25in
\noindent\textbf{Lab Assignments}:

\noindent There will be several lab assignments during the course. The main objective of these assignments is for students to employ the concepts and language features learned in class in a practical context.

%%%%%%%%%%%%%%%%%%%%%%%%%%%%%%%%%%%%%%%%%%%%%%%%%%%%%%%%%%%%
\vskip.25in
\noindent\textbf{Grading Policies}: \\

\noindent \textcolor{red}{\textbf{See the course webpage for policies regarding late lab assignments and plagiarism.}}

\begin{itemize}
\item Lab assignments: 30\%
\item Project: 10\%
\item Midterm Exam: 35\%
\item Final Exam: 35\%
\end{itemize}

\noindent The grading standard is:

\begin{itemize}
\item 90+ = A
\item 80+ = B
\item 70+ = C
\item 60+ = D
\item $<$60 = E
\end{itemize}



\end{document}